% !TEX program = xelatex
\documentclass{article}
\usepackage{fontspec}
\usepackage{xeCJK}
\setCJKmainfont{Noto Serif CJK JP} % 日本語フォント指定

\usepackage{amsthm}
\usepackage{amsmath}
\usepackage{amsfonts}

% 定理環境の定義
\theoremstyle{definition} % 定義用スタイル
\newtheorem{definition}{定義}[section] % 定義環境

\theoremstyle{plain} % 定理用スタイル
\newtheorem{theorem}{定理}[section]   % 定理環境
\newtheorem{lemma}[theorem]{補題}     % 補題環境
\newtheorem{proposition}[theorem]{命題} % 命題環境

\theoremstyle{remark} % 例・注記用スタイル
\newtheorem{example}{例}[section]     % 例環境
\newtheorem{remark}{注}[section]      % 注記環境

\begin{document}

\section{数学の基本}
\begin{definition}
\end{definition}

\begin{definition}[連続関数]
関数 $f: \mathbb{R} \to \mathbb{R}$ が点 $a$ で連続であるとは、
任意の $\epsilon > 0$ に対して、ある $\delta > 0$ が存在して、
$|x - a| < \delta$ ならば $|f(x) - f(a)| < \epsilon$ が成り立つことをいう。
\end{definition}

\begin{theorem}[中間値の定理]
関数 $f$ が閉区間 $[a, b]$ で連続で、$f(a) \neq f(b)$ ならば、
$f(a)$ と $f(b)$ の間の任意の値 $k$ に対して、
$f(c) = k$ となる $c \in (a, b)$ が存在する。
\end{theorem}

\begin{proof}
連続関数の性質より…(証明がここに入ります)
\end{proof}

\begin{example}
関数 $f(x) = x^2$ はすべての点で連続である。
\end{example}

\begin{lemma}[有用な補題]
ある条件が成り立つとき、別の条件も成り立つ。
\end{lemma}

\begin{proposition}[重要な命題]
ある主張が成り立つ。
\end{proposition}

\end{document}
