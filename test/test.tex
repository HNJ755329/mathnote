% !TEX program = xelatex
\documentclass[11pt,a4paper]{article}

\usepackage{fontspec}
\usepackage{xeCJK}

% ファミリー名で指定
\setmainfont{Noto Serif CJK JP}
\setCJKmainfont{Noto Serif CJK JP}


% ハイパーリンク(PDFのしおり用)
\usepackage{hyperref}

\title{XeLaTeX テスト文書}
\author{あなたの名前}
\date{\today}

\begin{document}

\maketitle

\section{はじめに}
これはXeLaTeXのテスト文書です。日本語とEnglishが混在しても問題なく処理できます。

\section{様々な要素のテスト}

\subsection{箇条書き}
\begin{itemize}
    \item 第一の項目
    \item 第二の項目
    \item \textbf{太字}や\textit{斜体}も使えます
\end{itemize}

\subsection{数式}
インライン数式: $E = mc^2$

別行立て数式:
\[
\int_{-\infty}^{\infty} e^{-x^2} dx = \sqrt{\pi}
\]

\subsection{表}
\begin{tabular}{|c|c|c|}
\hline
名前 & 年齢 & 職業 \\
\hline
山田太郎 & 25 & エンジニア \\
\hline
佐藤花子 & 30 & デザイナー \\
\hline
\end{tabular}

\section{結論}
XeLaTeXを使えば、日本語文書の作成がとても簡単になります!

\end{document}
